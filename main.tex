\documentclass{resume}

\usepackage[sfdefault]{ClearSans}

\begin{document}

% \fontfamily{ppl}\selectfont

% move dates to let of positon, Sept. instead of September, drop first year.

\noindent
\begin{minipage}[t][0pt]{\linewidth}
\begin{tabularx}{\linewidth}{@{}m{0.8\textwidth} m{0.2\textwidth}@{} }
{
    % \Large{\textbf{Tareq El Dandachi}}
    % \small{
    %     \clink{
    %         \href{mailto:tareq@mit.edu}{tareq@mit.edu} \textbf{·} Cambridge, USA \textbf{·} 
    %         % {\fontdimen2\font=0.75ex +1 2345 6789} 
    %         % \textbf{·} 
    %         % \href{https://github.com/tareqdandachi}{github.com/tareqdandachi}
    %         \href{https://tareqdandachi.github.io}{tareqdandachi.github.io}
    %     }
    % }
    
    \begin{tabularx}{\textwidth}{ X X }
    
    \Large{\textbf{Tareq ``Torque" El Dandachi}} &
    
    \small{
    \vspace{-19px}
        \clink{
            \rightline{ \href{mailto:tareq@mit.edu}{tareq@mit.edu} }\newline 
            \rightline{ \href{http://itstorque.com}{hi.itstorque.com} }
        }
    }
    
    \end{tabularx}
} & 
{
    \hfill
    % \includegraphics[width=2.8cm]{images/gr.png}
}
\end{tabularx}

\vspace{-20px}

\begin{center}
% \begin{tabularx}{\linewidth}{>{\hsize=0.5\hsize}X X}

\begin{tabularx}{\linewidth}{ p{6cm} X  }

% @{}*{2}{X}@{}
% left side %
{

    
    \csection{EDUCATION}{\small
    \vspace{10px}
    % Candidate for a double major in 
        % \begin{itemize}
            \frcontentdesc{Massachusetts Institute of Technology (MIT)}{
            M.Eng. in Electrical Engineering and Computer Science (2023)\newline
            B.S. in Electrical Engineering and Computer Science (2022)\newline
            B.S. in Mechanical Engineering and Quantum Information and Computation (2022)
            }{}{GPA: 4.9/5.0}
        % \end{itemize}
            
        
        \vspace{5px}
        
    }
    
    \csection{SKILLS}{\small
        % \begin{itemize}
        
        \def \skillSpacing {6px}
        
            \vspace{2px}
            
            \vspace{\skillSpacing}
        
            \textbf{Hardware \& Circuits} \newline
            {\footnotesize FPGA Design, Hardware Simulation, PCB Design, Processor Design, Nanoelectronics, Embedded Systems, Signal Processing, Electromagnetism, Solid State Circuits, SPICE Simulation}{}{}
            
            \vspace{\skillSpacing}
        
            \textbf{Software} \newline
            {\footnotesize Computational Photography, Computer Vision, Controls, Machine Learning, Security Research, Web Design}{}{}
        
            
            \vspace{\skillSpacing}
        
            \textbf{Physics} \newline
            {\footnotesize Circuit QED, Quantum Simulation, Optics, Superconductivity, Semi-Conductor Physics, Quantum Systems Control, Quantum Measurement, Nanophotonics}{}{}
            
            \vspace{\skillSpacing}
        
            \textbf{Mathematics} \newline
            {\footnotesize Linear Algebra, Group Theory, Complexity Theory, Information Theory, Calculus, Probability, Differential Equations}{}{}
            % \vspace{-5px}
            
            \vspace{\skillSpacing}
        
            \textbf{Mechanical Skills} \newline
            {\footnotesize Mill, Lathe, 3D printing, Robotics, Thermodynamics, Fluid Dynamics}{}{}
            
            \vspace{\skillSpacing}
        
            \textbf{Languages} \newline
            {\footnotesize \textbf{Software}: C, C++, Julia, Python, Swift, JavaScript, Ruby, Kotlin, MATLAB, Java, Objective-C, PHP, bash\newline
            \textbf{Hardware}: BlueSpec, Verik, SystemVerilog
            % \newline\textbf{Spoken}: English, French, Arabic
            }{}{}
            
            
            % \item \textbf{Languages} \newline
            % {\footnotesize Python, C++, MATLAB, JavaScript, PHP, Swift, CoffeeScript, Bash, Haskell, Objective-C, Logos, Java, Ruby, OpenQASM, Q\#.}{}{}
            % \vspace{-5px}
            % \item \textbf{Technologies} \newline
            % {\footnotesize Tensorflow, QuTIP, Qiskit, SLURM, OpenCV, ROS, Cross-Platform Apple Frameworks, Docker, PureData, ElectronJS, NodeJS, Theos, Jekyll,  jQuery.}{}{} % eventually u can add kLayout :)
            % % \item \textbf{Patterns \& Practices} \newline
            % % {\footnotesize Object Oriented Programming, Functional  Programming, CI \& CD, Microservices}
            % % \item \textbf{Project Management} \newline
            % % {\footnotesize Agile, Scrum, Google Bug Tracker, Google Workspace}
            % \vspace{-5px}
            % \item \textbf{Large-Scale Open Source Contributions} \newline
            % {\footnotesize Qiskit, OpenQASM syntax highlighters, iOS Security Exploits, Atom, matplotlib.}
        % \end{itemize}
    }

    \vspace{-200px}
    
} 
% end left side %
& 
% right side %
{

    \def \rightColVertSpacing {2px}

    \csection{EXPERIENCE}{\small
        \begin{itemize}
            % item 1 %
            % \item \frcontent{Dream Company}{Senior Software Engineer - San Fransisco, CA}{Co-designer and co-implementor of five successive generations of Google’s crawling indexing and query retrieval systems.}{August 1999 onwards}
            % item 2 %
            % \item \frcontent{Old Dream Company}{Senior Member of Technical Staff − Texas}{Designer and implementor of a system for retrieving and caching electronic commerce content including a crawler and custom full-text indexing system that allows flexible keyword searching of product information.}{February 1999 to August 1999}
            \item \frcontent{Superconducting Devices Researcher}{MIT Quantum Nanostructures and Nanofabrication}{
            Designing and fabricating superconducting electronics. Working on the application of theory to design
            better large-scale non-linear simulations of complex devices. Simulation driven redesigning 
            of various nanoelectronics - such as single photon detectors, imagers and time-to-digital
            converters. Developing new equipment and techniques for analyzing fabrication defects.
            }{May 2022 - Now}
            \vspace{\rightColVertSpacing}
            \item \frcontent{Electro-thermal Modelling of Superconducting Devices}{MIT Quantum Nanostructures and Nanofabrication}{
            Developed mathematical methods and implemented an electro-thermal model in Python to
            efficiently simulate superconducting wires and superconducting nanowire single photon detector (SNSPDs).
            }{May 2021 - May 2022}
            \vspace{\rightColVertSpacing}
            \item \frcontent{QuantumClifford.jl GPU Kernel Developer}{MIT Quantum Photonics Group}{
            Implemented fast GPU quantum stabilizer formalism simulations for a Julia quantum simulation package \textit{QuantumClifford.jl}
            }{Feb. 2022 - May 2022}
            \vspace{\rightColVertSpacing}
            \item \frcontent{Multiplexed optimal control of spin quantum memories}{MIT Quantum Photonics Group}{
        Built tensorflow optimizers that generate optimal microwave control pulses for diamond-based quantum computers with a web tool to view simulation results.
            Developed models simulating arbitrary arrangements of color centers and waveguides. Implemented optimal control theory to find control pulses that increase the number of qubits we can control on a diamond-based quantum computer by 3 orders of magnitude. Published a paper.
            }{Sept. 2020 - Oct. 2021}
            \vspace{\rightColVertSpacing}
            % \item \frcontent{Assistive Technology - MIT}{Executive and Director of Technology}{Organized hackathons, training sessions and lectures in the Assistive Technology space. Designed the website, code, logos and merchandise used by the group.}{September 2018 - Present}
            % \vspace{-5px}
            % \item \frcontent{MIT ESI Rapid Response Group (RRG)}{Web Designer}{Designed the new logo and website for MIT's Environmental Solutions Initiative new initiative RRG. Link to website: \clink{\href{https://rrg.mit.edu}{rrg.mit.edu}}}{June - Aug. 2020}
            % \vspace{\rightColVertSpacing}
            % \item \frcontent{Consulting Business Services (CBS)}{Java Developer}{Developed Java programs for CBS clients and implementations for services provided by Oracle to clients. \clink{\href{http://cbs.com.lb}{cbs.com.lb}}}{June - August 2019}
            % \vspace{-5px}
            % \item \frcontent{The Tech - MIT}{Director of Technology}{Maintained infrastructure running MIT's official newspaper: the physical servers, code the writers and publishers use and \clink{\href{https://thetech.com}{thetech.com}}}{September 2018 - October 2020}\vspace{-5px}
        \end{itemize}
    }
    
    
    
    \csection{PROJECTS}{\small
        \begin{itemize}
            \item \frcontentdesc{FPGA Depth Estimation using a Camera Array}{
            Programmed an FPGA to estimate depth information from two camera feeds.
            }{}{Jan. 2022
            % | \clink{\href{https://github.com/tareqdandachi/FPGA-Depth-Camera}{Link to github page.}}
            }
            \vspace{\rightColVertSpacing}
            
            \item \frcontentdesc{Eclipse\textit{ - glasses that modulate epileptic triggers}
}{
Developed and launched an alpha prototype with a team of product designers. I worked on sensing and modulation, user interaction, designing and performing EEG trials, coding in Microchip Studio and choosing PCB components.
}{}{Sept. - Dec. 2021  
% | \clink{\href{https://www.009hello.com/products/blue_brochure.pdf}{Link to product brochure}}
}
            \vspace{\rightColVertSpacing}
            
            \item \frcontentdesc{Non-Photorealistic Renderer}{Developed a C++ project that processes images and converts them into detailed multi-layered paintings with the ability to interpolate and incorporate design styles from a reference image.
            }{}{May 2020  
            % | \clink{\href{https://github.com/tareqdandachi/Non-Photorealistic-Renderer}{Link to github page.}}
            }
            \vspace{\rightColVertSpacing}
            \item \frcontentdesc{Computer Vision and LIDAR Based Obstacle Avoidance}{
            Programmed a self-driving car using ROS to race on a track while avoiding obstacles for MIT's Robotics Systems and Science. Leveraged image segmentation and classification to build the navigation space and path-find around obstacles. Implemented SLAM on LIDAR data with a team to assist navigation.
            }{}{Feb. - May 2020}
            \vspace{\rightColVertSpacing}
            % \item \frcontent{QASM Circuit Preview}{Designed a modification to IDEs that adds tools that help visualize your circuit and states as you code.}{}{January 2020}
            % \vspace{-5px}
            % \item \frcontent{OpenQASM Syntax Highlighter}{Programmed cross-IDE/platform syntax highlighters that highlight quantum assembly code. Strangeworks Inc. adopted the code into their projects.}{}{August 2020}
            % \vspace{-5px}
            % \item \frcontent{QuHackMan}{Collaborated with a team to design a quantum game of PacMan to help educate players on principles involving measurement and state evolution as part of a quantum hackathon and won an award for it. $\;\;\;\;\;\;\;\;\;$ $\;\;\;\;\;\;\;\;\;$ $\;\;\;\;\;\;\;\;\;$ \clink{\href{https://github.com/iQuHACK/QuhacMan}{Link to github.}}}{}{February 2020}
            % \vspace{-5px}
            % \item \frcontent{QuPong}{Designed a multiplayer quantum game of pong that involves building logic circuits to operate the player to teach the principles of quantum algorithmic thinking, Won 4th place in MIT's WebLab competition for it.$\;$ \clink{\href{https://github.com/weblab-class/tareqdandachi}{Link to github.}}}{}{January 2020}
            % \vspace{-60px}
            
        \end{itemize}
    }
    
    % \csection{ROMAN ANTHONY}{\small
    %     \begin{itemize}
    %         \item \frcontent{ANTHONY C ROMAN}{C ROMAN ANTHONY ROMAN ANTHONY C ROMAN ANTHONY}{}{ROMAN ANTHONY BIRTHDAY EVERYDAY EXCEPT ONe DAY}
    %     \end{itemize}
    % }
    
}


\end{tabularx}
\end{center}
\end{minipage}
\end{document}